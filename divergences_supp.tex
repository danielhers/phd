%
% File divergences_supp.tex

\chapter{Content Differences in Syntactic and Semantic Representation \\ Supplementary Material}

\section{Category Definitions}\label{sec:definitions}

\subsection{UCCA}

Table~\ref{tab:ucca} provides a concise description of the
categories used by the UCCA foundational layer.

\begin{table}[ht]
\centering
\footnotesize
\begin{tabular}{cp{1.6cm}p{12.5cm}}
\multicolumn{3}{c}{\bf Scene Elements}\\
P & {\bf Process} & The main relation of a Scene that evolves in time (usually an action or movement).\\
S & {\bf State} & The main relation of a Scene that does not evolve in time.\\
A & {\bf Participant} & A participant in a Scene in a broad sense (including locations, abstract entities and Scenes serving as arguments).\\
D & {\bf Adverbial} & A secondary relation in a Scene.\\
T & {\bf Time} & A temporal relation in a Scene.\\
\multicolumn{3}{c}{{\bf Elements of Non-Scene Units}}\\
C & {\bf Center} & Necessary for the conceptualization of the parent unit.\\
E & {\bf Elaborator} & A non-Scene relation which applies to a single Center.\\
N & {\bf Connector} & A non-Scene relation which applies to two or more Centers, highlighting a common feature.\\
R & {\bf Relator} & All other types of non-Scene relations. Two varieties: (1) Rs that relate a C to some super-ordinate relation, and
(2) Rs that relate two Cs pertaining to different aspects of the parent unit. \\
Q & {\bf Quantifier} & Describing the quantity or magnitude of something, or defines an entity as a group or a set (e.g., ``two'' or ``a group of'').\\
\multicolumn{3}{c}{\bf Inter-Scene Relations}\\
H & {\bf Parallel Scene} & A Scene linked to other Scenes by regular linkage (e.g., temporal, logical, purposive).\\
L & {\bf Linker} & A relation between two or more Hs (e.g., ``when'', ``if'', ``in order to'').\\
G & {\bf Ground} & A relation between the speech event and the uttered Scene (e.g., ``surprisingly'', ``in my opinion'').\\
\multicolumn{3}{c}{\bf Other}\\
F & {\bf Function} & Does not introduce a relation or participant. Required by the structural pattern it appears in.
\end{tabular}
\caption{The complete set of categories in UCCA's foundational layer.\label{tab:ucca}}
\end{table}

\subsection{Universal Dependencies}

Table~\ref{tab:ud} lists the full names of the
relation labels used by Universal Dependencies v2.

\begin{table}[ht]
\centering
\footnotesize
\begin{tabular}{ll}
\texttt{acl} & clausal modifier of noun (adjectival clause) \\
\texttt{advcl} & adverbial clause modifier \\
\texttt{advmod} & adverbial modifier \\
\texttt{amod} & adjectival modifier \\
\texttt{appos} & appositional modifier \\
\texttt{aux} & auxiliary \\
\texttt{case} & case marking \\
\texttt{cc} & coordinating conjunction \\
\texttt{ccomp} & clausal complement \\
\texttt{clf} & classifier \\
\texttt{compound} & compound \\
\texttt{conj} & conjunct \\
\texttt{cop} & copula \\
\texttt{csubj} & clausal subject \\
\texttt{dep} & unspecified dependency \\
\texttt{det} & determiner \\
\texttt{discourse} & discourse element \\
\texttt{dislocated} & dislocated elements \\
\texttt{expl} & expletive \\
\texttt{fixed} & fixed multiword expression \\
\texttt{flat} & flat multiword expression \\
\texttt{goeswith} & goes with \\
\texttt{iobj} & indirect object \\
\texttt{list} & list \\
\texttt{mark} & marker \\
\texttt{nmod} & nominal modifier \\
\texttt{nsubj} & nominal subject \\
\texttt{nummod} & numeric modifier \\
\texttt{obj} & object \\
\texttt{obl} & oblique nominal \\
\texttt{orphan} & orphan \\
\texttt{parataxis} & parataxis \\
\texttt{punct} & punctuation \\
\texttt{reparandum} & overridden disfluency \\
\texttt{root} & root \\
\texttt{vocative} & vocative \\
\texttt{xcomp} & open clausal complement
\end{tabular}
\caption{UD v2 relations.\label{tab:ud}}
\end{table}

